
\section{Transmissions from mobile source (R/V Sharp)}


\subsection{21:11GMT-21:27GMT}
Radar image (figure??) shows at about 21:15GMT, a small ISW package
reaches the VLA, which can also be verified on the temperature
plots(figure??). Modes 3, 4 and 5 are starting to increase and
oscillate around 21:22GMT, while the total intensity still shows
little fluctuation.



\subsection{21:41GMT-21:57GMT}
ISW totally intersects the acoustic path. The fast oscillation in
the previous period of the signal is replaced by single, strong
focusing/defocusing. The fluctuation of the total intensity is about
8dB. Both VLA and HLA show strong fluctuation. HLA shows slopes
during all transmission - means ISW is in the path. Also, note that
the slope changes during 6 minutes. From the modal decomposition
plots, it can be shown the focusing mainly happens to mode 1 and 2,
when mode 3, 4, and 5 show little increase in intensity.

\subsection{22:11GMT-22:27GMT}
ISW moving across the entire acoustic path. At about 22:13, there is
a small focusing event with 5dB increase in total intensity, which
is shown on all the modes but more pronounced on mode 1 and 2
(??dB). It also shows on the HLA with a weak but consistent slope
across all the hydrophones. At about 22:21-22:25, a longer-lasting
focusing is recorded, which only happens at the water column below
35m. Mode 2 is the biggest contributor among all modes. On HLA, the
focusing stops showing about 200m away from the center of SHARK.

\subsection{22:41GMT-22:57GMT}
The acoustic path starts to enter the tail of the ISW package, when
the wave fronts are more irregular, and we lost the radar coverage
at the receiver. The angle between the acoustic path and the leading
front is about $15^o$ - the small angle assumption no long holds.
Peaked at 22:51GMT, a major focusing event are clearly recorded with
8dB fluctuation in total intensity. It crosses all the hydrophones
on the VLA and HLA. Modal decomposition shows the mode 1, 2 and 3
are the major contributors.

\subsection{23:11GMT-23:27GMT}
R/V Sharp is transmitting by a very big angle (~$30^o$). The
fluctuation is very small (~2dB). Mode 1 shows a peak at 23:14GMT,
while mode 2 and 4 start to grow at 23:13GMT. Mode coupling?

%\subsection{Intensity variation and angular dependence}
