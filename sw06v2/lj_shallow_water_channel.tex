
\chapter{Shallow water acoustic channel }
%Shallow water acoustic channel has some different properties when
%compared to aerial radio channel.

In this chapter,we will describe the effect of rough sea surface on
acoustic wave propagation. We start with an overview of acoustic
wave equations and solutions, followed by an introduction to ray
theory and some unique properties of shallow water acoustic channel.
The focus will be on the effect of boundaries on acoustic wave
propagation.

The following symbols and notations will be used:

$\vec{r}=$equilibrium position of a fluid element
\begin{equation}
\vec{r}=x\hat{x}+y\hat{y}+z\hat{z}
\end{equation}
($\hat{x}$,$\hat{y}$ and $\hat{z}$ are the unit vectors in the $x$,
$y$ and $z$ directions, respectively.)

$\vec{\xi}=$ particle displacement of a fluid element from its
equilibrium prosition
\begin{equation}
\vec{\xi}=\xi_{x}\hat{x}+\xi_{y}\hat{y}+\xi_{z}\hat{z}
\end{equation}

$\vec{u}=$particle velocity of a fluid element
\begin{equation}
\vec{u}=\frac{\partial{\vec{\xi}}}{\partial{t}}=u_x{\hat{x}}+u_y{\hat{y}}+u_z{\hat{z}}
\end{equation}

$\rho=$instantaneous density at $(x,y,z)$

$\rho_0=$equilibrium density at $(x,y,z)$

$s=$condensation at $(x,y,z)$

\begin{equation}
s=(\rho-\rho_0)/\rho_0
\end{equation}

$\rho-\rho_0=\rho_0s=$acoustic density at $(x,y,z)$

$\cal{P}=$instantaneous pressure at $(x,y,z)$

$\cal{P} _0=$equilibrium pressure at $(x,y,z)$

$p$ acoustic pressure at $(x,y,z)$
\begin{equation}
p=\cal{P}-\cal{P}_0
\end{equation}

$c=$thermodynamic speed of sound of the fluid

$\Phi=$velocity potential of the wave
\begin{equation}
\vec{u}=\nabla\Phi
\end{equation}

$T_k=$temperature in kelvins(K)

$T=$temperature in degree Celsius (or centigrade)($^\circ C$)

\begin{equation}
T+273.15=T_k
\end{equation}

\section{Acoustic Wave Equation and Solutions}
\subsection{Equation of state}

\subsection{Equation of continuity}
\subsection{Linear wave equation}
\section{Ray Theory}

\section{Acoustic Wave in a WaveGuide}
\section{Boundary and Volume}
\section{Propagation Issues}
\subsection{Refraction}
\subsection{Reflection}
\subsection{Multipath}
\section{Our Contributions}
